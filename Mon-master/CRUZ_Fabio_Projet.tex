% Options for packages loaded elsewhere
\PassOptionsToPackage{unicode}{hyperref}
\PassOptionsToPackage{hyphens}{url}
\PassOptionsToPackage{dvipsnames,svgnames,x11names}{xcolor}
%
\documentclass[
  11pt,
  a4paperpaper,
  onecolumn]{article}

\usepackage{amsmath,amssymb}
\usepackage{iftex}
\ifPDFTeX
  \usepackage[T1]{fontenc}
  \usepackage[utf8]{inputenc}
  \usepackage{textcomp} % provide euro and other symbols
\else % if luatex or xetex
  \usepackage{unicode-math}
  \defaultfontfeatures{Scale=MatchLowercase}
  \defaultfontfeatures[\rmfamily]{Ligatures=TeX,Scale=1}
\fi
\usepackage[]{mathdesign}
\ifPDFTeX\else  
    % xetex/luatex font selection
\fi
% Use upquote if available, for straight quotes in verbatim environments
\IfFileExists{upquote.sty}{\usepackage{upquote}}{}
\IfFileExists{microtype.sty}{% use microtype if available
  \usepackage[]{microtype}
  \UseMicrotypeSet[protrusion]{basicmath} % disable protrusion for tt fonts
}{}
\makeatletter
\@ifundefined{KOMAClassName}{% if non-KOMA class
  \IfFileExists{parskip.sty}{%
    \usepackage{parskip}
  }{% else
    \setlength{\parindent}{0pt}
    \setlength{\parskip}{6pt plus 2pt minus 1pt}}
}{% if KOMA class
  \KOMAoptions{parskip=half}}
\makeatother
\usepackage{xcolor}
\usepackage[left=2cm,top=1.5cm,right=2cm,bottom=1.5cm]{geometry}
\setlength{\emergencystretch}{3em} % prevent overfull lines
\setcounter{secnumdepth}{-\maxdimen} % remove section numbering
% Make \paragraph and \subparagraph free-standing
\makeatletter
\ifx\paragraph\undefined\else
  \let\oldparagraph\paragraph
  \renewcommand{\paragraph}{
    \@ifstar
      \xxxParagraphStar
      \xxxParagraphNoStar
  }
  \newcommand{\xxxParagraphStar}[1]{\oldparagraph*{#1}\mbox{}}
  \newcommand{\xxxParagraphNoStar}[1]{\oldparagraph{#1}\mbox{}}
\fi
\ifx\subparagraph\undefined\else
  \let\oldsubparagraph\subparagraph
  \renewcommand{\subparagraph}{
    \@ifstar
      \xxxSubParagraphStar
      \xxxSubParagraphNoStar
  }
  \newcommand{\xxxSubParagraphStar}[1]{\oldsubparagraph*{#1}\mbox{}}
  \newcommand{\xxxSubParagraphNoStar}[1]{\oldsubparagraph{#1}\mbox{}}
\fi
\makeatother


\providecommand{\tightlist}{%
  \setlength{\itemsep}{0pt}\setlength{\parskip}{0pt}}\usepackage{longtable,booktabs,array}
\usepackage{calc} % for calculating minipage widths
% Correct order of tables after \paragraph or \subparagraph
\usepackage{etoolbox}
\makeatletter
\patchcmd\longtable{\par}{\if@noskipsec\mbox{}\fi\par}{}{}
\makeatother
% Allow footnotes in longtable head/foot
\IfFileExists{footnotehyper.sty}{\usepackage{footnotehyper}}{\usepackage{footnote}}
\makesavenoteenv{longtable}
\usepackage{graphicx}
\makeatletter
\newsavebox\pandoc@box
\newcommand*\pandocbounded[1]{% scales image to fit in text height/width
  \sbox\pandoc@box{#1}%
  \Gscale@div\@tempa{\textheight}{\dimexpr\ht\pandoc@box+\dp\pandoc@box\relax}%
  \Gscale@div\@tempb{\linewidth}{\wd\pandoc@box}%
  \ifdim\@tempb\p@<\@tempa\p@\let\@tempa\@tempb\fi% select the smaller of both
  \ifdim\@tempa\p@<\p@\scalebox{\@tempa}{\usebox\pandoc@box}%
  \else\usebox{\pandoc@box}%
  \fi%
}
% Set default figure placement to htbp
\def\fps@figure{htbp}
\makeatother
% definitions for citeproc citations
\NewDocumentCommand\citeproctext{}{}
\NewDocumentCommand\citeproc{mm}{%
  \begingroup\def\citeproctext{#2}\cite{#1}\endgroup}
\makeatletter
 % allow citations to break across lines
 \let\@cite@ofmt\@firstofone
 % avoid brackets around text for \cite:
 \def\@biblabel#1{}
 \def\@cite#1#2{{#1\if@tempswa , #2\fi}}
\makeatother
\newlength{\cslhangindent}
\setlength{\cslhangindent}{1.5em}
\newlength{\csllabelwidth}
\setlength{\csllabelwidth}{3em}
\newenvironment{CSLReferences}[2] % #1 hanging-indent, #2 entry-spacing
 {\begin{list}{}{%
  \setlength{\itemindent}{0pt}
  \setlength{\leftmargin}{0pt}
  \setlength{\parsep}{0pt}
  % turn on hanging indent if param 1 is 1
  \ifodd #1
   \setlength{\leftmargin}{\cslhangindent}
   \setlength{\itemindent}{-1\cslhangindent}
  \fi
  % set entry spacing
  \setlength{\itemsep}{#2\baselineskip}}}
 {\end{list}}
\usepackage{calc}
\newcommand{\CSLBlock}[1]{\hfill\break\parbox[t]{\linewidth}{\strut\ignorespaces#1\strut}}
\newcommand{\CSLLeftMargin}[1]{\parbox[t]{\csllabelwidth}{\strut#1\strut}}
\newcommand{\CSLRightInline}[1]{\parbox[t]{\linewidth - \csllabelwidth}{\strut#1\strut}}
\newcommand{\CSLIndent}[1]{\hspace{\cslhangindent}#1}

\usepackage{fancyhdr}
\pagestyle{fancy}
\fancyhead[c]{Research proposal}
\fancyfoot[L]{MADELHIS: Epistémologie, histoire des sciences et des techniques}
\fancyfoot[c]{}
\fancyfoot[R]{\thepage}
\makeatletter
\@ifpackageloaded{caption}{}{\usepackage{caption}}
\AtBeginDocument{%
\ifdefined\contentsname
  \renewcommand*\contentsname{Table of contents}
\else
  \newcommand\contentsname{Table of contents}
\fi
\ifdefined\listfigurename
  \renewcommand*\listfigurename{List of Figures}
\else
  \newcommand\listfigurename{List of Figures}
\fi
\ifdefined\listtablename
  \renewcommand*\listtablename{List of Tables}
\else
  \newcommand\listtablename{List of Tables}
\fi
\ifdefined\figurename
  \renewcommand*\figurename{Figure}
\else
  \newcommand\figurename{Figure}
\fi
\ifdefined\tablename
  \renewcommand*\tablename{Table}
\else
  \newcommand\tablename{Table}
\fi
}
\@ifpackageloaded{float}{}{\usepackage{float}}
\floatstyle{ruled}
\@ifundefined{c@chapter}{\newfloat{codelisting}{h}{lop}}{\newfloat{codelisting}{h}{lop}[chapter]}
\floatname{codelisting}{Listing}
\newcommand*\listoflistings{\listof{codelisting}{List of Listings}}
\makeatother
\makeatletter
\makeatother
\makeatletter
\@ifpackageloaded{caption}{}{\usepackage{caption}}
\@ifpackageloaded{subcaption}{}{\usepackage{subcaption}}
\makeatother

\usepackage{bookmark}

\IfFileExists{xurl.sty}{\usepackage{xurl}}{} % add URL line breaks if available
\urlstyle{same} % disable monospaced font for URLs
\hypersetup{
  pdftitle={Rethinking the role and responsibility of Engineering in the Anthropocene},
  pdfauthor={Fabio A. CRUZ SANCHEZ},
  colorlinks=true,
  linkcolor={blue},
  filecolor={Maroon},
  citecolor={Blue},
  urlcolor={blue},
  pdfcreator={LaTeX via pandoc}}


\title{Rethinking the role and responsibility of Engineering in the
Anthropocene}
\usepackage{etoolbox}
\makeatletter
\providecommand{\subtitle}[1]{% add subtitle to \maketitle
  \apptocmd{\@title}{\par {\large #1 \par}}{}{}
}
\makeatother
\subtitle{Insights from Ethical and Historical Perspectives}
\author{Fabio A. CRUZ SANCHEZ}
\date{March 22, 2025}

\begin{document}
\maketitle


\thispagestyle{fancy}

Plastic waste
contamination\textsuperscript{\citeproc{ref-de-la-torre2021}{1}},
climate change\textsuperscript{\citeproc{ref-stoddard2021}{2}},
biodiversity loss\textsuperscript{\citeproc{ref-hermoso2022}{3}} are
majors markups of what is recently disscused as the Anthropocene
era\textsuperscript{\citeproc{ref-steffen2018}{4},\citeproc{ref-steffen2011}{5}}.
The anthropocene frames the humans not only as biological but as
geological force acknowledging the new status of humanity given the
different indicators in the natural ecosystems that are impacting the
stability of the earth system.

The globalized mass manufacturing paradigm have played a major role not
only as motor for the economic development, but also the transgression
of the planetary
boundaries\textsuperscript{\citeproc{ref-ONeill2018}{6},\citeproc{ref-raworth2017}{7}}.
The mass manufacturing socio-technical systems is understood as a deep
transition. Manufacturing systems requires materials as well as human
and physical capital to produce goods. This co-evolution remains not to
solve recurring issues of social inequality in connection to unequal
access to healthcare, energy, water, food, mobility, security, finance,
education, and communication. Even if the importance of manufacturing as
the heart of an economy has not changed, the way of producing goods and
the setup of the location start to change dramatically.

The circular economy concept entry in the
policy\textsuperscript{\citeproc{ref-EC2015}{8}},
industrial\textsuperscript{\citeproc{ref-EllenMacArthurFoundation2015}{9}}
and
scientific\textsuperscript{\citeproc{ref-nobre2021}{10}--\citeproc{ref-Schoggl2020}{12}}
arenas as an umbrella concept, but also as a contested
one\textsuperscript{\citeproc{ref-CalistoFriant2020}{13}--\citeproc{ref-corvellec2021}{15}},
aiming to change the societal conciousness that the ecological systems
have nearly endless capacity to provide resources and adsorb wastes.
However, \textbf{Engineering science needs to integrate that the
externalities\textsuperscript{\citeproc{ref-zhen2021}{16}} of human
activites' impacts on the earth systems since the fuzzy front-end phase
of the innovation process.}

Since 2014, I have been working on the validation of \emph{distributed
recycling for additive manufacturing
(DRAM)}\textsuperscript{\citeproc{ref-CruzSanchez2020}{17},\citeproc{ref-CruzSanchez2017}{18}}
as technological option for plastic waste recycling. Likewise, I have
been working on the design of the pertinent closed-loop supply
chain\textsuperscript{\citeproc{ref-Santander2020}{19}}, considering the
applicable sustainability indicators based on the scientific literature
an including the `maker mouvement' approaches (fablabs, tiers lieux,
hackspaces) in the equation.

Nevertheless, I realize that many questions remains open in the
scientific and pedagogical aspects:

\begin{itemize}
\tightlist
\item
  If the technical systems of '\emph{design global, manufacture local'}
  continue to be democratized, there is a risk of a rebound effect
  leading to increased waste. How can ethical and historical
  perspectives help us understand and address this paradox?
\item
  What political, economic, social, legal, and ethical barriers make it
  difficult to transition away from a growth-driven economic paradigm?
\item
  Can peer production contribute to the democratization of both
  technology and society?
\item
  What role do engineering and sustainability sciences play in shaping a
  post-growth economy? and in what extent the engineering schools in
  France need to include these reflections in their curricula?
\end{itemize}

Certainly, there are more relevant and precise questions from ethical
and political perspectives. However, the main idea is to continue
learning the key critical thinking to put in perspective the
(techno-push) engineering solutions that emerges every day for the
understanding of the soio-ecological issues.

\newpage

\subsection*{References}\label{references}
\addcontentsline{toc}{subsection}{References}

\phantomsection\label{refs}
\begin{CSLReferences}{0}{0}
\bibitem[\citeproctext]{ref-de-la-torre2021}
\CSLLeftMargin{1. }%
\CSLRightInline{De-la-Torre, G. E., Dioses-Salinas, D. C.,
Pizarro-Ortega, C. I. \& Santillán, L.
\href{https://doi.org/10.1016/j.scitotenv.2020.142216}{New plastic
formations in the {Anthropocene}}. \emph{Science of The Total
Environment} \textbf{754}, 142216 (2021).}

\bibitem[\citeproctext]{ref-stoddard2021}
\CSLLeftMargin{2. }%
\CSLRightInline{Stoddard, I. \emph{et al.}
\href{https://doi.org/10.1146/ANNUREV-ENVIRON-012220-011104}{Three
{Decades} of {Climate Mitigation}: {Why Haven}'t {We Bent} the {Global
Emissions Curve}?}
\emph{https://doi.org/10.1146/annurev-environ-012220-011104}
\textbf{46}, 653--689 (2021).}

\bibitem[\citeproctext]{ref-hermoso2022}
\CSLLeftMargin{3. }%
\CSLRightInline{Hermoso, V. \emph{et al.}
\href{https://doi.org/10.1016/J.ENVSCI.2021.10.028}{The {EU Biodiversity
Strategy} for 2030: {Opportunities} and challenges on the path towards
biodiversity recovery}. \emph{Environmental Science \& Policy}
\textbf{127}, 263--271 (2022).}

\bibitem[\citeproctext]{ref-steffen2018}
\CSLLeftMargin{4. }%
\CSLRightInline{Steffen, W. \emph{et al.}
\href{https://doi.org/10.1073/pnas.1810141115}{Trajectories of the
{Earth System} in the {Anthropocene}}. \emph{Proceedings of the National
Academy of Sciences} \textbf{115}, 8252--8259 (2018).}

\bibitem[\citeproctext]{ref-steffen2011}
\CSLLeftMargin{5. }%
\CSLRightInline{Steffen, W., Grinevald, J., Crutzen, P. \& McNeill, J.
\href{https://doi.org/10.1098/rsta.2010.0327}{The {Anthropocene}:
Conceptual and historical perspectives}. \emph{Philosophical
Transactions of the Royal Society A: Mathematical, Physical and
Engineering Sciences} \textbf{369}, 842--867 (2011).}

\bibitem[\citeproctext]{ref-ONeill2018}
\CSLLeftMargin{6. }%
\CSLRightInline{O'Neill, D. W., Fanning, A. L., Lamb, W. F. \&
Steinberger, J. K. \href{https://doi.org/10.1038/s41893-018-0021-4}{A
good life for all within planetary boundaries}. \emph{Nature
Sustainability} \textbf{1}, 88--95 (2018).}

\bibitem[\citeproctext]{ref-raworth2017}
\CSLLeftMargin{7. }%
\CSLRightInline{Raworth, K.
\href{https://doi.org/10.1016/S2542-5196(17)30028-1}{A {Doughnut} for
the {Anthropocene}: Humanity's compass in the 21st century}. \emph{The
Lancet Planetary Health} \textbf{1}, e48--e49 (2017).}

\bibitem[\citeproctext]{ref-EC2015}
\CSLLeftMargin{8. }%
\CSLRightInline{European Commision.
\href{https://doi.org/10.1017/CBO9781107415324.004}{Summary for
{Policymakers}}. in \emph{Climate {Change} 2013 - {The Physical Science
Basis}} (ed. Intergovernmental Panel on Climate Change) vol. 614 1--30
({Cambridge University Press}, {Cambridge}, 2015).}

\bibitem[\citeproctext]{ref-EllenMacArthurFoundation2015}
\CSLLeftMargin{9. }%
\CSLRightInline{Ellen MacArthur Foundation. Growth within: A circular
economy vision for a competitive europe. \emph{Ellen MacArthur
Foundation} 100 (2015) doi:\href{https://doi.org/Article}{Article}.}

\bibitem[\citeproctext]{ref-nobre2021}
\CSLLeftMargin{10. }%
\CSLRightInline{Nobre, G. C. \& Tavares, E.
\href{https://doi.org/10.1016/j.jclepro.2021.127973}{The quest for a
circular economy final definition: {A} scientific perspective}.
\emph{Journal of Cleaner Production} \textbf{314}, 127973 (2021).}

\bibitem[\citeproctext]{ref-Kirchherr2017}
\CSLLeftMargin{11. }%
\CSLRightInline{Kirchherr, J., Reike, D. \& Hekkert, M.
\href{https://doi.org/10.1016/j.resconrec.2017.09.005}{Conceptualizing
the circular economy: {An} analysis of 114 definitions}.
\emph{Resources, Conservation and Recycling} \textbf{127}, 221--232
(2017).}

\bibitem[\citeproctext]{ref-Schoggl2020}
\CSLLeftMargin{12. }%
\CSLRightInline{Schöggl, J.-P., Stumpf, L. \& Baumgartner, R. J.
\href{https://doi.org/10.1016/j.resconrec.2020.105073}{The narrative of
sustainability and circular economy - {A} longitudinal review of two
decades of research}. \emph{Resources, Conservation and Recycling}
\textbf{163}, 105073 (2020).}

\bibitem[\citeproctext]{ref-CalistoFriant2020}
\CSLLeftMargin{13. }%
\CSLRightInline{Calisto Friant, M., Vermeulen, W. J. V. \& Salomone, R.
\href{https://doi.org/10.1016/j.resconrec.2020.104917}{A typology of
circular economy discourses: {Navigating} the diverse visions of a
contested paradigm}. \emph{Resources, Conservation and Recycling}
\textbf{161}, 104917 (2020).}

\bibitem[\citeproctext]{ref-rodl2022}
\CSLLeftMargin{14. }%
\CSLRightInline{Rödl, M. B., Åhlvik, T., Bergeå, H., Hallgren, L. \&
Böhm, S. \href{https://doi.org/10.1016/J.JCLEPRO.2022.132144}{Performing
the {Circular} economy: {How} an ambiguous discourse is managed and
maintained through meetings}. \emph{Journal of Cleaner Production}
\textbf{360}, 132144 (2022).}

\bibitem[\citeproctext]{ref-corvellec2021}
\CSLLeftMargin{15. }%
\CSLRightInline{Corvellec, H., Stowell, A. F. \& Johansson, N. Critiques
of the circular economy. \emph{Journal of Industrial Ecology} (2021)
doi:\href{https://doi.org/10.1111/JIEC.13187}{10.1111/JIEC.13187}.}

\bibitem[\citeproctext]{ref-zhen2021}
\CSLLeftMargin{16. }%
\CSLRightInline{Zhen, H., Gao, W., Yuan, K., Ju, X. \& Qiao, Y.
\href{https://doi.org/10.1016/j.ecoser.2021.101323}{Internalizing
externalities through net ecosystem service analysis\textendash{{A}}
case study of greenhouse vegetable farms in {Beijing}}. \emph{Ecosystem
Services} \textbf{50}, 101323 (2021).}

\bibitem[\citeproctext]{ref-CruzSanchez2020}
\CSLLeftMargin{17. }%
\CSLRightInline{Cruz Sanchez, F. A., Boudaoud, H., Camargo, M. \&
Pearce, J. M.
\href{https://doi.org/10.1016/j.jclepro.2020.121602}{Plastic recycling
in additive manufacturing: {A} systematic literature review and
opportunities for the circular economy}. \emph{Journal of Cleaner
Production} \textbf{264}, 121602 (2020).}

\bibitem[\citeproctext]{ref-CruzSanchez2017}
\CSLLeftMargin{18. }%
\CSLRightInline{Cruz Sanchez, F. A., Boudaoud, H., Hoppe, S. \& Camargo,
M. \href{https://doi.org/10.1016/j.addma.2017.05.013}{Polymer recycling
in an open-source additive manufacturing context: {Mechanical} issues}.
\emph{Additive Manufacturing} \textbf{17}, 87--105 (2017).}

\bibitem[\citeproctext]{ref-Santander2020}
\CSLLeftMargin{19. }%
\CSLRightInline{Santander, P., Cruz Sanchez, F. A., Boudaoud, H. \&
Camargo, M.
\href{https://doi.org/10.1016/j.resconrec.2019.104531}{{Closed loop
supply chain network for local and distributed plastic recycling for 3D
printing: a MILP-based optimization approach}}. \emph{Resources,
Conservation and Recycling} \textbf{154}, 104531 (2020).}

\end{CSLReferences}




\end{document}
